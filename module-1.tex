\section{Module 1 - Understanding the Computer}
\subsection{Big Ideas}
\begin{enumerate}
  \item \textbf{Hardware} - CPU, I/O, Memory, etc. that are used to run your
    computer 
  \item \textbf{Kernel} - The middleman between software applications you run
    (game, browser, etc.) and the actual hardware of the computer. It manages
    system resources, ensuring everything works together smoothly and
    efficiently
  \begin{itemize}
    \item Handles tasks like running programs, accessing files, and connecting
      to devices like printers and keyboards. 
  \end{itemize}
  \item \textbf{Application Code} - Instructions written in a programming
    language that tell a software program (or simply app) what to do, processing
    data and performing specific tasks for the user (VSCode, compiler, Chrome,
    shell)
  \item Often there are many details in an application you do not know yet and
    many details that aren't important at all in an application.
    \textbf{Abstraction} allows us to often ignore aspects we don't need to really worry about or
    about or dive into 
  \item All code runs on \textbf{hardware}
  \item Compiler, application, CPU, PC, ALU, OS, memory, disk, I/O are all
    various hardware/software that work together to run a computer program!
  \item The \textbf{OS} (Operating System) has a major role 
  \begin{itemize}
    \item Switches between running programs 
    \item Interfaces with many parts of the hardware like memory and other
      devices
  \end{itemize}
\end{enumerate}
\subsection{CLI - Terminal/Shell Overview}
There are many ways to interface with a computer. The two most common ones being
the \textbf{GUI} (Graphical User Interface - where you click things!) and \textbf{CLI} (Command 
Line Interface - where you type stuff in a terminal!) \\\\ 
The CLI is very fast and efficient and is often preferred by many when dealing
with file systems! Using a terminal running a shell is often great to 
\begin{itemize}
  \item Navigate your files!
  \item Run applications! (Python, your compiler - GCC, ssh/scp)
\end{itemize}
\subsubsection{Simple Vocab}
\begin{enumerate}
  \item \textbf{Terminal/Command Line} - The interface where you run commands!
  \item \textbf{Shell} - The software that executes commands in the command
    line!
  \item \textbf{Directory (Folder)} - A folder in the file system 
  \item \textbf{Path} - The location of a file or folder 
\end{enumerate}
\subsubsection{Basic Commands}
\begin{itemize}
  \item \texttt{ls} - lists all files and directories in your working directory
    (Use -al flag after ls to include hidden files (-a) and permissions (-l))
  \item \texttt{cd} - changes the directory (a.k.a. folder!). Use \texttt{.} to
    indicate the current path. Use \texttt{..} to go back one directory in a
    path!
  \item \texttt{pwd} - Print the current working directory. Check what file path
    you are in. Good to know where you are in your file system!
\end{itemize}
\subsection{Running a C/C++ file}
\begin{enumerate}
  \item Compile $\rightarrow$ turns program into an executable
  \item Run the executable file $\rightarrow$ ran on CPU!
\end{enumerate}
Every C/C++ file needs to be compiled! \\\\ 
For example, suppose there are three files, \texttt{a.c}, \texttt{a.h} (a header
file for \texttt{a.c}), and \texttt{b.c}. To compile!
\begin{enumerate}
  \item Compile each .c file listed
  \item Link together .o files (compiled machine code files!)
  \item If there are many, use make (a.ka. create a Makefile!) to easily compile
    all dependencies and run program!
\end{enumerate}
When you update your written code, you need to recompile it into a new
executable to see the change!!!!
