\section{Module 2 - C (not C++)} 
\textbf{C} is a very popular \textbf{low level programming language}. Low level
languages are very beneficial because they give you lots of control over your
computer. This means that you as the programmer have direct control over a
computer's memory and processing. 
\subsection{How is C different from C++?}
\begin{enumerate}
    \item No Templates
    \item No Classes
    \item No \texttt{new} or \texttt{delete}
    \item No pass-by-reference
    \item No standard C++ library (strings, vectors, maps, cin/cout, etc.)
\end{enumerate}
\subsection{Structs in C}
Is the code below in \textbf{Example 1.1} valid C code?
\begin{lstlisting}[language=C, caption={An invalid C struct}]
struct food {
  int amount;
  int age;

  int can_eat() {
    if(amount > 0 && age < 10) {
      return 1;
    }
    return 0;
  }
};
\end{lstlisting}
\textbf{NO!}
Structs in C are created in a different way than C++!
\begin{enumerate}
  \item Create a struct with ONLY member variables
  \item Functions \textbf{must} work with a pointer to an instance of that
    struct
\end{enumerate}
\newpage
\textbf{Example 1.2} shows a correct way to declare the struct in C, following
the two rules above 
\begin{lstlisting}[language=C, caption={Corrected C struct from Example 1.1}]
typedef struct food {
  int amount;
  int age;
} food;

int can_eat(food *self) {
  if(self->age < 10) {
    return 1;
  }
  return 0;
}
\end{lstlisting}
\subsection{Revisiting \texttt{.} (dot operator) and \texttt{->} (arrow
operator) from C++}
\texttt{.} (\textbf{dot operator}) - Used to access members from a \textbf{object} \\
\texttt{->} (\textbf{arrow operator}) - Used to access members from a
\textbf{pointer} \\\\
Suppose I took \textbf{Example 1.2} and instead of having function
\verb|can_eat| take
in a pointer to struct \texttt{food}, it takes in an object instead. The code
would now look like one shown in \textbf{Example 1.3}
\begin{lstlisting}[language=C, caption={Example 1.2, but function takes in
object instead of pointer}]
typedef struct food {
  int amount;
  int age;
} food;

int can_eat(food self) {
  if(self.age < 10) {
    return 1;
  }
  return 0;
}
\end{lstlisting}
Notice that the only difference is literally using the dot operator instead of
the arrow operator! The dot operator was used since that is the appropriate
operator to access the member of an object!
\subsection{\texttt{malloc} and \texttt{free}}
\verb|malloc| - equivalent to C++ \verb|new|. Allocations size bytes on the
\textbf{heap} and returns a \textbf{pointer} to that memory location on the heap 
\begin{center}
    \begin{minipage}{0.8\textwidth}
        \begin{lstlisting}[language=C]
void *malloc(size_t size); 
        \end{lstlisting}
    \end{minipage}
\end{center}
\verb|malloc| takes in the total size of memory to allocate, which is typically
done via calculating the size of a type (using \verb|sizeof| operator), see
\textbf{Example 1.4}
\begin{lstlisting}[language=C, caption={Using \texttt{malloc} in C}]
  int* ptr = malloc(sizeof(int)) 
\end{lstlisting}
where \verb|sizeof(int)| $= 4$ bytes (or equivalently 32 bits) \\\\ 
\verb|free| - equivalent to C++ \verb|delete| 
